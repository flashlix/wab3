%
% Dokumentation SWE Energy 
%

\documentclass[fontsize=12pt, paper=a4, headinclude, twoside=false, parskip=half+, pagesize=auto, numbers=noenddot, plainheadsepline, open=right, toc=listof, toc=bibliography]{scrreprt}
% PDF-Kompression
\pdfminorversion=5
\pdfobjcompresslevel=1
% Allgemeines
\usepackage[automark]{scrpage2} % Kopf- und Fuߟzeilen
\usepackage{amsmath,marvosym} % Mathesachen
\usepackage[T1]{fontenc} % Ligaturen, richtige Umlaute im PDF
\usepackage[latin1]{inputenc}% UTF8-Kodierung f�r Umlaute usw
\usepackage[section]{placeins}% Bildkorrektur
% Schriften
\usepackage{mathpazo} % Palatino f�r Mathemodus
%\usepackage{mathpazo,tgpagella} % auch sehr sch�ne Schriften
\usepackage{setspace} % Zeilenabstand
%\onehalfspacing % 1,5 Zeilen
% Schriften-Gr�ߟen
\setkomafont{chapter}{\Huge\rmfamily} % ܜberschrift der Ebene
\setkomafont{section}{\Large\rmfamily}
\setkomafont{subsection}{\large\rmfamily}
\setkomafont{subsubsection}{\large\rmfamily}
\setkomafont{chapterentry}{\large\rmfamily} % ܜberschrift der Ebene in Inhaltsverzeichnis
\setkomafont{descriptionlabel}{\bfseries\rmfamily} % f�r description Umgebungen
\setkomafont{captionlabel}{\small\bfseries}
\setkomafont{caption}{\small}
% Sprache: Deutsch
\usepackage[ngerman]{babel} % Silbentrennung
% PDF
\usepackage[ngerman, pdfauthor={Felix Michalek}, pdfauthor={Felix Michalek}, pdftitle={Vorlage f�r LaTeX}, breaklinks=true,baseurl={}]{hyperref}
\usepackage[final]{microtype} % mikrotypographische Optimierungen
\usepackage{url}
\usepackage{pdflscape} % einzelne Seiten drehen k�nnen
% Tabellen
\usepackage{multirow} % Tabellen-Zellen �ber mehrere Zeilen
\usepackage{multicol} % mehre Spalten auf eine Seite
\usepackage{tabularx} % F�r Tabellen mit vorgegeben Gr�ߟen
\usepackage{longtable} % Tabellen �ber mehrere Seiten
\usepackage{array}
% Bibliographie
\usepackage{bibgerm} % Umlaute in BibTeX
% Tabellen
\usepackage{multirow} % Tabellen-Zellen �ber mehrere Zeilen
\usepackage{multicol} % mehre Spalten auf eine Seite
\usepackage{tabularx} % F�r Tabellen mit vorgegeben Gr�ߟen
\usepackage{array}
\usepackage{float}
% Bilder
\usepackage{graphicx} % Bilder
\usepackage{color} % Farben
\graphicspath{{Bilder/}}
%\DeclareGraphicsExtensions{.pdf,.png,.jpg} % bevorzuge pdf-Dateien
\usepackage{subfigure} % mehrere Abbildungen nebeneinander/�bereinander
\newcommand{\subfigureautorefname}{\figurename} % um \autoref auch f�r subfigures benutzen
\usepackage[all]{hypcap} % Beim Klicken auf Links zum Bild und nicht zu Caption gehen
% Bildunterschrift
\setcapindent{0em} % kein Einr�cken der Caption von Figures und Tabellen
\setcapwidth[c]{0.9\textwidth}
\setlength{\abovecaptionskip}{0.2cm} % Abstand der zwischen Bild- und Bildunterschrift
% Quellcode
\usepackage{listings} % f�r Formatierung in Quelltexten
\definecolor{grau}{gray}{0.25}
\lstset{
	extendedchars=true,
	basicstyle=\tiny\ttfamily,
	%basicstyle=\footnotesize\ttfamily,
	tabsize=2,
	keywordstyle=\textbf,
	commentstyle=\color{grau},
	stringstyle=\textit,
	numbers=left,
	numberstyle=\tiny,
	% f�r sch�nen Zeilenumbruch
	breakautoindent  = true,
	breakindent      = 2em,
	breaklines       = true,
	postbreak        = ,
	prebreak         = \raisebox{-.8ex}[0ex][0ex]{\Righttorque},
}
% linksb�ndige Fuߟboten
\deffootnote{1.5em}{1em}{\makebox[1.5em][l]{\thefootnotemark}}

\typearea{14} % typearea am Schluss berechnen lassen, damit die Einstellungen oben ber�cksichtigt werden
% f�r autoref von Gleichungen in itemize-Umgebungen
\makeatletter
\newcommand{\saved@equation}{}
\let\saved@equation\equation
\def\equation{\@hyper@itemfalse\saved@equation}
\makeatother 



% Eigene Befehle %%%%%%%%%%%%%%%%%%%%%%%%%%%%%%%%%%%%%%%%%%%%%%%%%5
% Matrix
\newcommand{\mat}[1]{
      {\textbf{#1}}
}
\newcommand{\todo}[1]{
      {\colorbox{red}{ TODO: #1 }}
}
\newcommand{\todotext}[1]{
      {\color{red} TODO: #1} \normalfont
}
\newcommand{\info}[1]{
      {\colorbox{blue}{ (INFO: #1)}}
}
% Hinweis auf Programme in Datei
\newcommand{\datei}[1]{
      {\ttfamily{#1}}
}
\newcommand{\code}[1]{
      {\ttfamily{#1}}
}

% bild einfach
\newcommand{\bild}[3]{
	\begin{figure}[!htb]
 		\centering
 		\includegraphics[width=0.8\textwidth]							
 						{#1/#1_Seite_#2_Bild_0001.jpg}
 		\caption{#3 (\cite{#1} S. #2)}
 		\label{img.#1_#2}
 	\end{figure}
 	\FloatBarrier
}
% bild einfach zwei
\newcommand{\bildtwo}[3]{
	\begin{figure}[!htb]
 		\centering
 		\includegraphics[width=0.8\textwidth]							
 						{#1/#1_Seite_#2_Bild_0002.jpg}
 		\caption{#3 (\cite{#1} S. #2)}
 		\label{img.#1_#2}
 	\end{figure}
 	\FloatBarrier
}
% bild mit defnierter Breite einf�gen
\newcommand{\bilde}[4]{
  \begin{figure}[!hbt]
    \centering
      \vspace{1ex}
      \includegraphics[width=#2]{Bilder/#1}
      \caption[#4]{\label{img.#1} #3}
      \vspace{1ex}
  \end{figure}
}
% bild mit defnierter Breite einf�gen
\newcommand{\bilded}[5]{
  \begin{figure}[!hbt]
    \centering
      \vspace{1ex}
      \includegraphics[width=#2,#5]{Bilder/#1}
      \caption[#4]{\label{img.#1} #3}
      \vspace{1ex}
  \end{figure}
}
% bild mit eigener Breite
\newcommand{\bilda}[3]{
  \begin{figure}[!hbt]
    \centering
      \vspace{1ex}
      \includegraphics{Bilder/#1}
      \caption[#3]{\label{img.#1} #2}
      \vspace{1ex}
  \end{figure}
}


% Bild todo
\newcommand{\bildt}[2]{
  \begin{figure}[!hbt]
    \begin{center}
      \vspace{2ex}
	      \includegraphics[width=6cm]{images/todobild}
      %\caption{\label{#1} \color{red}{ TODO: #2}}
      \caption{\label{#1} \todotext{#2}}
      %{\caption{\label{#1} {\todo{#2}}}}
      \vspace{2ex}
    \end{center}
  \end{figure}
}


\title{WAB ged�hns auf SWE Vorlage\\Software Engeneering Pr�fungsvorleistung Energiesimulation}
\author{Thomas Melcher\\Felix Michalek\\Steffen Witte\\Robert Gei�ler\\Sebastian Ludwig\\Skadi Passek\\ hier ist meine erste �nderung}
\date{\today}

\begin{document}
	\maketitle
	\tableofcontents
	
	
	\chapter{Projekttreiber}
		\section{Zweck des Projekts}
			\subsection{Inhalt}
				Realisierung eines komplexen Softwareprojekts dessen Ergebnis eine Pr�fungsvorleistung ist. Ziel ist es eine Simulation zu erstellen die f�r eine Menge von Energieverbrauchern, Energieerzeugern und Speichern �berpr�ft ob der Energiehaushalt ausgeglichen ist.
			\subsection{Motivation}
				Das Projekt ist eine Aufgabe im Fach Softwareengeneering des Studiums der Wirtschaftsinformatik und ist Voraussetzung f�r die Modulpr�fung. 
 			\subsection{Form}
 				Als Ergebnis stehen am Ende die Energie Simulation als kommentierter Quelltext und verschiedene Entwickler/Projekt Dokumentationen in der Form von UMLs und verschiedenen Diagrammen.
		\section{Kunden und andere Beteiligte}
			Prof. Dr.-Ing. Sabine Wieland
			wieland@hft-leipzig.de

			Dipl.-Ing. (FH) Karsten Hain 
			hain@hft-leipzig.de


	\chapter{Vorgegebene Randbedingungen f�r das Projekt}
		\section{Entwicklungsumgebung}
			\subsection{Beschreibung}
				Die Energiesimulation ist eine Webapplikation. Sie ist Betriebssystem- und Browserunabh�ngig. 
			\subsection{Motivation}
				Die Simulation ist eine Webapplikation um Installationen f�r den Nutzer zu vermeiden.
		\section{Externe Software}
			Ben�tigt zum Ausf�hren wird ein aktueller HTML5 f�higer Browser.
		\section{Zeitliche Eingrenzung}
			Abgabetermin ist 2 Wochen vor Vorlesungsende.

	\chapter{Relevante Fakten und Annahmen}
		\section{Quellen}
			\cite{1, 2, 3, 4, 5, 6, 7, 8, 9, 0}
		\section{Annahmen}
			Es gab folgende Annahmen:
			\begin{itemize}
      			\item Blockheizkraftwerke sind in der Realit�t Gaskraftwerke deren Abw�rme in Siedlungen genutzt werden. Da in unserer Simulation keine W�rme betrachtet wird sind sie f�r uns nicht von belangen.
      			\item �lkraftwerke stellen weniger als 1\% in der deutschen Energieversorgung dar und sind deshalb irrelevant f�r die Simulation. Anstatt des �lkraftwerks wurde ein Wasserkraftwerk hinzugef�gt
      			\item Sonnenlicht verh�lt sich nicht zuf�llig sondern Zeitabh�ngig.
      			\item Kraftwerke m�ssen nicht gebaut wurden, es wird nur die Vorbereitungszeit f�r die Anbindung ans Stromnetz ber�cksichtigt
      			\item Energiespeicher spei�en ins gesamte Stromnetz ein und sind nicht exklusiv f�r den Haushalt der sie beinhaltet
      			\item Energiespeicher haben eine Kapazit�t, eine Entlade- und Laderate und berechnen sich nicht prozentual von der aktuellen Gesamtenergieproduktion wie in der Aufgabenstellung beschrieben wurde. Das von uns angenommene Verhalten entsprich eher der Realit�t weil Energiespeicher eine feste Kapazit�t haben.
      			\item Die Region f�r die Simulation ist Deutschland
      			\item Haushalte verbrauchen nicht statisch, sondern tageszeitabh�ngig
      			\item Die Simulationsgeschwindigkeit ist w�hlbar
   			\end{itemize}
	\chapter{Datenmodelle und Diagramme}
		\section{Klassendiagramm}
			\bilded{Class}{0.9\textwidth}{Klassendiagramm}{Klassendiagramm}{angle={270}}
		\section{Sequenzdiagramm}
			\bilded{Sequence}{0.8\textwidth}{Sequenzdiagramm}{Sequenzdiagramm}{angle={270}}
			\newpage
		\section{Zustandsdiagramm}
			\bilde{State}{0.6\textwidth}{Zustandsdiagramm}{Zustandsdiagramm}
			\newpage
		\section{Aktivit�tsdiagramm}
			\bilde{Activity}{0.8\textwidth}{Aktivit�tsdiagramm}{Aktivit�tsdiagramm}
			\newpage
		\section{Anwendungsfalldiagramm}
			\bilded{UseCase}{0.6\textwidth}{Anwendungsfalldiagramm}{Klassendiagramm}{angle={270}}
		
	\chapter{Funktionaler Umfang}
			\bilde{Aufgabenstellung}{0.7\textwidth}{Aufgabenstellung}{Aufgabenstellung}
		\section{Quellcode}
			Die erarbeiteten Quelldateien und UML Diagramme befinden sich in den gleichnamigen Ordnern. In dem Ordner ''Energieprojekt/Quell Daten/energie.zip/energy'' ist der von uns erarbeitete Quellcode in Form von ''.coffee'' Dateien enthalten.
	\appendix
	\listoffigures
	\begin{thebibliography}{999}
		\bibitem 1 de.wikipedia.org/wiki/Liste\_deutscher\_Kraftwerke
 		\bibitem 2 ag-energiebilanzen.de/[...]\_brd\_stromerzeugung1990\_2012.pdf
 		\bibitem 3 de.wikipedia.org/wiki/Liste\_der\_Kernreaktoren\_in\_Deutschland
 		\bibitem 4 www.zeit.de/wissen/umwelt/2011-05/standby-atomkraftwerk-text/seite-1
 		\bibitem 5 de.wikipedia.org/wiki/Kohlekraftwerk
 		\bibitem 6 commons.wikimedia.org/wiki/File:PV-Norddeutschland-2008-Monatsdarstellung.svg
 		\bibitem 7 commons.wikimedia.org/wiki/File:Photovoltaics\_change\_of\_production\_during\_day\_and\_year.png
 		\bibitem 8 de.wikipedia.org/wiki/Liste\_von\_Wasserkraftwerken\_in\_Deutschland
 		\bibitem 9 www.wir-ernten-was-wir-saeen.de/energiespiel/content/stromangebot-kraftwerke
 		\bibitem 0 www.vdi.de/[...]/CO2-Emissionen[...]Stromerzeugung\_01.pdf

	\end{thebibliography}
\end{document}

%
% EOF
%

