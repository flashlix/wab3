%
% Dokumentation SWE Energy 
%

\input{praeambel.tex}


\title{WAB ged�hns auf SWE Vorlage\\Software Engeneering Pr�fungsvorleistung Energiesimulation}
\author{Thomas Melcher\\Felix Michalek\\Steffen Witte\\Robert Gei�ler\\Sebastian Ludwig\\Skadi Passek\\ hier ist meine erste �nderung}
\date{\today}

\begin{document}
	\maketitle
	\tableofcontents
	
	
	\chapter{Projekttreiber}
		\section{Zweck des Projekts}
			\subsection{Inhalt}
				Realisierung eines komplexen Softwareprojekts dessen Ergebnis eine Pr�fungsvorleistung ist. Ziel ist es eine Simulation zu erstellen die f�r eine Menge von Energieverbrauchern, Energieerzeugern und Speichern �berpr�ft ob der Energiehaushalt ausgeglichen ist.
			\subsection{Motivation}
				Das Projekt ist eine Aufgabe im Fach Softwareengeneering des Studiums der Wirtschaftsinformatik und ist Voraussetzung f�r die Modulpr�fung. 
 			\subsection{Form}
 				Als Ergebnis stehen am Ende die Energie Simulation als kommentierter Quelltext und verschiedene Entwickler/Projekt Dokumentationen in der Form von UMLs und verschiedenen Diagrammen.
		\section{Kunden und andere Beteiligte}
			Prof. Dr.-Ing. Sabine Wieland
			wieland@hft-leipzig.de

			Dipl.-Ing. (FH) Karsten Hain 
			hain@hft-leipzig.de


	\chapter{Vorgegebene Randbedingungen f�r das Projekt}
		\section{Entwicklungsumgebung}
			\subsection{Beschreibung}
				Die Energiesimulation ist eine Webapplikation. Sie ist Betriebssystem- und Browserunabh�ngig. 
			\subsection{Motivation}
				Die Simulation ist eine Webapplikation um Installationen f�r den Nutzer zu vermeiden.
		\section{Externe Software}
			Ben�tigt zum Ausf�hren wird ein aktueller HTML5 f�higer Browser.
		\section{Zeitliche Eingrenzung}
			Abgabetermin ist 2 Wochen vor Vorlesungsende.

	\chapter{Relevante Fakten und Annahmen}
		\section{Quellen}
			\cite{1, 2, 3, 4, 5, 6, 7, 8, 9, 0}
		\section{Annahmen}
			Es gab folgende Annahmen:
			\begin{itemize}
      			\item Blockheizkraftwerke sind in der Realit�t Gaskraftwerke deren Abw�rme in Siedlungen genutzt werden. Da in unserer Simulation keine W�rme betrachtet wird sind sie f�r uns nicht von belangen.
      			\item �lkraftwerke stellen weniger als 1\% in der deutschen Energieversorgung dar und sind deshalb irrelevant f�r die Simulation. Anstatt des �lkraftwerks wurde ein Wasserkraftwerk hinzugef�gt
      			\item Sonnenlicht verh�lt sich nicht zuf�llig sondern Zeitabh�ngig.
      			\item Kraftwerke m�ssen nicht gebaut wurden, es wird nur die Vorbereitungszeit f�r die Anbindung ans Stromnetz ber�cksichtigt
      			\item Energiespeicher spei�en ins gesamte Stromnetz ein und sind nicht exklusiv f�r den Haushalt der sie beinhaltet
      			\item Energiespeicher haben eine Kapazit�t, eine Entlade- und Laderate und berechnen sich nicht prozentual von der aktuellen Gesamtenergieproduktion wie in der Aufgabenstellung beschrieben wurde. Das von uns angenommene Verhalten entsprich eher der Realit�t weil Energiespeicher eine feste Kapazit�t haben.
      			\item Die Region f�r die Simulation ist Deutschland
      			\item Haushalte verbrauchen nicht statisch, sondern tageszeitabh�ngig
      			\item Die Simulationsgeschwindigkeit ist w�hlbar
   			\end{itemize}
	\chapter{Datenmodelle und Diagramme}
		\section{Klassendiagramm}
			\bilded{Class}{0.9\textwidth}{Klassendiagramm}{Klassendiagramm}{angle={270}}
		\section{Sequenzdiagramm}
			\bilded{Sequence}{0.8\textwidth}{Sequenzdiagramm}{Sequenzdiagramm}{angle={270}}
			\newpage
		\section{Zustandsdiagramm}
			\bilde{State}{0.6\textwidth}{Zustandsdiagramm}{Zustandsdiagramm}
			\newpage
		\section{Aktivit�tsdiagramm}
			\bilde{Activity}{0.8\textwidth}{Aktivit�tsdiagramm}{Aktivit�tsdiagramm}
			\newpage
		\section{Anwendungsfalldiagramm}
			\bilded{UseCase}{0.6\textwidth}{Anwendungsfalldiagramm}{Klassendiagramm}{angle={270}}
		
	\chapter{Funktionaler Umfang}
			\bilde{Aufgabenstellung}{0.7\textwidth}{Aufgabenstellung}{Aufgabenstellung}
		\section{Quellcode}
			Die erarbeiteten Quelldateien und UML Diagramme befinden sich in den gleichnamigen Ordnern. In dem Ordner ''Energieprojekt/Quell Daten/energie.zip/energy'' ist der von uns erarbeitete Quellcode in Form von ''.coffee'' Dateien enthalten.
	\appendix
	\listoffigures
	\begin{thebibliography}{999}
		\bibitem 1 de.wikipedia.org/wiki/Liste\_deutscher\_Kraftwerke
 		\bibitem 2 ag-energiebilanzen.de/[...]\_brd\_stromerzeugung1990\_2012.pdf
 		\bibitem 3 de.wikipedia.org/wiki/Liste\_der\_Kernreaktoren\_in\_Deutschland
 		\bibitem 4 www.zeit.de/wissen/umwelt/2011-05/standby-atomkraftwerk-text/seite-1
 		\bibitem 5 de.wikipedia.org/wiki/Kohlekraftwerk
 		\bibitem 6 commons.wikimedia.org/wiki/File:PV-Norddeutschland-2008-Monatsdarstellung.svg
 		\bibitem 7 commons.wikimedia.org/wiki/File:Photovoltaics\_change\_of\_production\_during\_day\_and\_year.png
 		\bibitem 8 de.wikipedia.org/wiki/Liste\_von\_Wasserkraftwerken\_in\_Deutschland
 		\bibitem 9 www.wir-ernten-was-wir-saeen.de/energiespiel/content/stromangebot-kraftwerke
 		\bibitem 0 www.vdi.de/[...]/CO2-Emissionen[...]Stromerzeugung\_01.pdf

	\end{thebibliography}
\end{document}

%
% EOF
%

