\documentclass[fontsize=12pt, paper=a4, headinclude, twoside=false, parskip=half+, pagesize=auto, numbers=noenddot, plainheadsepline, open=right, toc=listof, toc=bibliography]{scrreprt}
% PDF-Kompression
\pdfminorversion=5
\pdfobjcompresslevel=1
% Allgemeines
\usepackage[automark]{scrpage2} % Kopf- und Fuߟzeilen
\usepackage{amsmath,marvosym} % Mathesachen
\usepackage[T1]{fontenc} % Ligaturen, richtige Umlaute im PDF
\usepackage[latin1]{inputenc}% UTF8-Kodierung f�r Umlaute usw
\usepackage[section]{placeins}% Bildkorrektur
% Schriften
\usepackage{mathpazo} % Palatino f�r Mathemodus
%\usepackage{mathpazo,tgpagella} % auch sehr sch�ne Schriften
\usepackage{setspace} % Zeilenabstand
%\onehalfspacing % 1,5 Zeilen
% Schriften-Gr�ߟen
\setkomafont{chapter}{\Huge\rmfamily} % ܜberschrift der Ebene
\setkomafont{section}{\Large\rmfamily}
\setkomafont{subsection}{\large\rmfamily}
\setkomafont{subsubsection}{\large\rmfamily}
\setkomafont{chapterentry}{\large\rmfamily} % ܜberschrift der Ebene in Inhaltsverzeichnis
\setkomafont{descriptionlabel}{\bfseries\rmfamily} % f�r description Umgebungen
\setkomafont{captionlabel}{\small\bfseries}
\setkomafont{caption}{\small}
% Sprache: Deutsch
\usepackage[ngerman]{babel} % Silbentrennung
% PDF
\usepackage[ngerman, pdfauthor={Felix Michalek}, pdfauthor={Felix Michalek}, pdftitle={Vorlage f�r LaTeX}, breaklinks=true,baseurl={}]{hyperref}
\usepackage[final]{microtype} % mikrotypographische Optimierungen
\usepackage{url}
\usepackage{pdflscape} % einzelne Seiten drehen k�nnen
% Tabellen
\usepackage{multirow} % Tabellen-Zellen �ber mehrere Zeilen
\usepackage{multicol} % mehre Spalten auf eine Seite
\usepackage{tabularx} % F�r Tabellen mit vorgegeben Gr�ߟen
\usepackage{longtable} % Tabellen �ber mehrere Seiten
\usepackage{array}
% Bibliographie
\usepackage{bibgerm} % Umlaute in BibTeX
% Tabellen
\usepackage{multirow} % Tabellen-Zellen �ber mehrere Zeilen
\usepackage{multicol} % mehre Spalten auf eine Seite
\usepackage{tabularx} % F�r Tabellen mit vorgegeben Gr�ߟen
\usepackage{array}
\usepackage{float}
% Bilder
\usepackage{graphicx} % Bilder
\usepackage{color} % Farben
\graphicspath{{Bilder/}}
%\DeclareGraphicsExtensions{.pdf,.png,.jpg} % bevorzuge pdf-Dateien
\usepackage{subfigure} % mehrere Abbildungen nebeneinander/�bereinander
\newcommand{\subfigureautorefname}{\figurename} % um \autoref auch f�r subfigures benutzen
\usepackage[all]{hypcap} % Beim Klicken auf Links zum Bild und nicht zu Caption gehen
% Bildunterschrift
\setcapindent{0em} % kein Einr�cken der Caption von Figures und Tabellen
\setcapwidth[c]{0.9\textwidth}
\setlength{\abovecaptionskip}{0.2cm} % Abstand der zwischen Bild- und Bildunterschrift
% Quellcode
\usepackage{listings} % f�r Formatierung in Quelltexten
\definecolor{grau}{gray}{0.25}
\lstset{
	extendedchars=true,
	basicstyle=\tiny\ttfamily,
	%basicstyle=\footnotesize\ttfamily,
	tabsize=2,
	keywordstyle=\textbf,
	commentstyle=\color{grau},
	stringstyle=\textit,
	numbers=left,
	numberstyle=\tiny,
	% f�r sch�nen Zeilenumbruch
	breakautoindent  = true,
	breakindent      = 2em,
	breaklines       = true,
	postbreak        = ,
	prebreak         = \raisebox{-.8ex}[0ex][0ex]{\Righttorque},
}
% linksb�ndige Fuߟboten
\deffootnote{1.5em}{1em}{\makebox[1.5em][l]{\thefootnotemark}}

\typearea{14} % typearea am Schluss berechnen lassen, damit die Einstellungen oben ber�cksichtigt werden
% f�r autoref von Gleichungen in itemize-Umgebungen
\makeatletter
\newcommand{\saved@equation}{}
\let\saved@equation\equation
\def\equation{\@hyper@itemfalse\saved@equation}
\makeatother 



% Eigene Befehle %%%%%%%%%%%%%%%%%%%%%%%%%%%%%%%%%%%%%%%%%%%%%%%%%5
% Matrix
\newcommand{\mat}[1]{
      {\textbf{#1}}
}
\newcommand{\todo}[1]{
      {\colorbox{red}{ TODO: #1 }}
}
\newcommand{\todotext}[1]{
      {\color{red} TODO: #1} \normalfont
}
\newcommand{\info}[1]{
      {\colorbox{blue}{ (INFO: #1)}}
}
% Hinweis auf Programme in Datei
\newcommand{\datei}[1]{
      {\ttfamily{#1}}
}
\newcommand{\code}[1]{
      {\ttfamily{#1}}
}

% bild einfach
\newcommand{\bild}[3]{
	\begin{figure}[!htb]
 		\centering
 		\includegraphics[width=0.8\textwidth]							
 						{#1/#1_Seite_#2_Bild_0001.jpg}
 		\caption{#3 (\cite{#1} S. #2)}
 		\label{img.#1_#2}
 	\end{figure}
 	\FloatBarrier
}
% bild einfach zwei
\newcommand{\bildtwo}[3]{
	\begin{figure}[!htb]
 		\centering
 		\includegraphics[width=0.8\textwidth]							
 						{#1/#1_Seite_#2_Bild_0002.jpg}
 		\caption{#3 (\cite{#1} S. #2)}
 		\label{img.#1_#2}
 	\end{figure}
 	\FloatBarrier
}
% bild mit defnierter Breite einf�gen
\newcommand{\bilde}[4]{
  \begin{figure}[!hbt]
    \centering
      \vspace{1ex}
      \includegraphics[width=#2]{Bilder/#1}
      \caption[#4]{\label{img.#1} #3}
      \vspace{1ex}
  \end{figure}
}
% bild mit defnierter Breite einf�gen
\newcommand{\bilded}[5]{
  \begin{figure}[!hbt]
    \centering
      \vspace{1ex}
      \includegraphics[width=#2,#5]{Bilder/#1}
      \caption[#4]{\label{img.#1} #3}
      \vspace{1ex}
  \end{figure}
}
% bild mit eigener Breite
\newcommand{\bilda}[3]{
  \begin{figure}[!hbt]
    \centering
      \vspace{1ex}
      \includegraphics{Bilder/#1}
      \caption[#3]{\label{img.#1} #2}
      \vspace{1ex}
  \end{figure}
}


% Bild todo
\newcommand{\bildt}[2]{
  \begin{figure}[!hbt]
    \begin{center}
      \vspace{2ex}
	      \includegraphics[width=6cm]{images/todobild}
      %\caption{\label{#1} \color{red}{ TODO: #2}}
      \caption{\label{#1} \todotext{#2}}
      %{\caption{\label{#1} {\todo{#2}}}}
      \vspace{2ex}
    \end{center}
  \end{figure}
}